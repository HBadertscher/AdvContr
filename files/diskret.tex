%TODO: discrete-time controllers: stability
%TODO: add digital implementation

% % % % % % % % % % % % % % % % % % % %
\section{Discrete-time Plant Models}

\subsection{ZOH-discretization of plant models}
Starting from a continuous-time state-space model $(A,B,C,D)$, the ZOH-discretization is derived.
In continuous-time,
\[
    x(t) = \Phi(t) \cdot x(0) + \Gamma(t) \cdot u(0)
\]
where
\begin{align*}
    \Phi(t) &= \mathcal{L}^{-1} \left( (sI-A)^{-1} \right) = e^{At} &
    & \text{and} &
    \Gamma(t) = \int_0^t \phi(t) \cdot B \cdot dt = \mathcal{L}^{-1} \left((sI-A)^{-1} \cdot B \cdot \frac{1}{s}\right)
\end{align*}

For any sampling time instant, the discrete-time state-space representation is
\begin{align*}
    x_{k+1} &= \Phi(T_s) \cdot x_k + \Gamma(T_s) \cdot u_k \\
    y_k &= C \cdot x_k + D \cdot u_k
\end{align*}

\subsection{ZOH-discretization of transfer functions}
Given a plant model $P(s)$, the ZOH-discretization is
\[
    P_{ZOH}(z) = \frac{z-1}{z}\mathcal{Z}_s\left\{ \frac{P(s)}{s} \right\}
\]

% % % % % % % % % % % % % % % % % % % %
\section{Modelling of discrete-time systems}
A discrete transfer function is gien by
\[
    \frac{Y(z)}{U(z)} = \frac{b_n + b_{n-1}z^{-1}+ \dots + b_0z^{-n}}{1 + a_{n-1}z^{-1} + \dots + a_0}
\]

A discrete-time state-space representation is a matrix difference equation:
\begin{align*}
    x_{k+1} &= F \cdot x_k + G \cdot u_k \\
    y_k &= H \cdot x_k + J \cdot u_k
\end{align*}
which can be written in a transfer function form given by
\[
    P_{ZOH}(z) = H \cdot (zI - F)^{-1} \cdot G + J
\]

\subsection{Properties of sampled systems}

\paragraph{Causality}A transfer function with a higher degree of the numerator is \emph{noncausal}
and is not realizable.

\paragraph{Stability}The stability can be determined using the relation $z = e^{sT_s}$.
The left-half plane in the $s$-domain corresponds to the area inside the unit circle in the $z$-domain.

\paragraph{Bode plot}The Bode plot can be drawn by setting $z = e^{j \omega T_s}$.
Above $\omega = \frac{\omega_s}{2}$, the frequency response is repeated with a period of $\omega_s$
due to the periodicity of $e^{j \omega T_s}$.

\paragraph{DC Gain}The final value of a step response, i.e. the DC gain is determined by
$\lim_{z \to 1}(G(z))$.

% % % % % % % % % % % % % % % % % % % %
\section{Discrete-time Controllers}
Controller discretization is basically the approximation of the integrator $\frac{1}{s}$.

\subsection{Euler approximation}
The Euler approximation uses the recursive definition of an integrator which leads to the transformation
\begin{align*}
    y(kT)) = y((k-1)T) + u(kT) T &&
    \frac{1}{s} \quad \Leftrightarrow \quad \frac{T \cdot z}{z-1}
\end{align*}

\subsection{Tustin approximation (Bilinear transform)}
When the sampling frequency is close to the system dynamics, the following transformation is used
\[
    s = \frac{2}{T} \cdot \frac{z - 1}{z + 1}
\]
This is also the most suitable discretization method for feedback controllers.

\paragraph{Prewarping}
The Tustin approximation warps the frequency response of the system.
A frequency $\omega$ is moved to $\frac{2}{T} \cdot \tan(\frac{\omega T_s}{2})$.
The following procedure can be used to prewarp the frequencies:
\begin{enumerate}
    \item Find discrete transfer function $C_{ZOH}(z)$ from continuous $C(s)$.
    \item Prewarp frequencies by applying the inverse Tustin approximation:
        \[
            z = \frac{2 + T_s \cdot w}{2 - T_s \cdot w}
        \]
        which leads to the pseud-continuous time-domain $w$.
    \item Design a controller $\tilde{C}(w)$ in the $w$-domain.
    \item Apply inverse bilinear transform to $\tilde{C}_d(w)$ to obtain controller $C_d(z)$:
        \[
            w = \frac{2}{T_s} \cdot \frac{z - 1}{z + 1}
        \]
\end{enumerate}

\subsection{Zero order hold discretization}
The controller is discretized under the assumption that the input signal is created by a zero order hold.

\subsection{Pole and zero matching}
The discrete-time poles and zeros are calculated by $z = e^{s T}$.
The discrete-time transfer function can be obtained from these poles and zeros.
Then the gain is chosen to provide a good approximation at a desired frequency.
This method is used when poles and zeros have to be matched exactly, e.g. in notch filters.

\subsection{Sampling Time}
Rules of thumb to determine the sampling frequency are
\begin{align*}
    \frac{\omega_s}{\omega_d} > 10 && \text{or} &&
    \frac{\omega_s}{\omega_{al}} > 10
\end{align*}
where $\omega_d$ is the closed loop bandwidth (crossing over frequency) and $\omega_{al}$ is
the bandwidth of the anti-aliasing filter
